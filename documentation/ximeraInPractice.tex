\documentclass{amsart}
\usepackage{multicol}
\usepackage{kmath,kerkis}
\usepackage{multirow}
\usepackage{fancyvrb}


\begin{document}
\pagenumbering{gobble}
\title{XIMERA: Markup in Practice} %%%% CHANGE PER COURSE
\maketitle

\section{Writing Exercises}

Perhaps the most basic form of interactive is the ability to turn a
static exercise into an interactive exercise.

\begin{Verbatim}[frame=single,numbers=left]
  \begin{exercise}
    Compute $2+2$
    \begin{answser}
      $17$
    \end{answer}
  \end{exercise}
\end{Verbatim}

\begin{Verbatim}[frame=single,numbers=left]
  \begin{exercise}
    Compute $2'+2'$
    \begin{answser}
      $17$ feet
    \end{answer}
  \end{exercise}
\end{Verbatim}


\begin{Verbatim}[frame=single,numbers=left]
  \begin{exercise}
    Compute $3\cdot 5$
    \begin{hint}
      $3\cdot 5$ represents $3$ groups of $5$
      \pause
      So we compute
      \[
      5+5+5
      \]
      \pause
      $=15$. 
    \end{hint}
    \begin{answser}
      $15$
    \end{answer}
  \end{exercise}
\end{Verbatim}


\begin{Verbatim}[frame=single,numbers=left]
\begin{exercise}[\var{n}={integer from 0 to 10, 3}]
Compute $2^\var{n}$.
\begin{answser}
$\evaluate{2^\var{n}}$ 
\end{answer}
\end{exercise}
\end{Verbatim}



\begin{Verbatim}[frame=single,numbers=left]
\begin{exercise}[\var{n}={integer from 1 to 10 and -1 to -10, 3}]
Compute $1/\var{n}$.
\begin{answser}
$\evaluate{1/\var{n}}$ 
\end{answer}
\end{exercise}
\end{Verbatim}

\end{document}
